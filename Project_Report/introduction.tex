%!TeX root = main.tex
\chapter{Introduction}

\section{Overview}
In the rapidly evolving job market, the recruitment process has become increasingly rigorous, often involving multiple rounds of technical assessments, coding challenges, and behavioral interviews. For many candidates, especially students and fresh graduates, the lack of accessible, personalized, and constructive feedback is a significant hurdle. \textbf{AceIt} is designed to bridge this gap by providing an intelligent, on-demand interview preparation platform.

\section{Motivation}
Traditional methods of interview preparation, such as mock interviews with peers or mentors, are often resource-constrained and may not provide objective or consistent feedback. Furthermore, self-study using static resources lacks the interactive and adaptive nature of real interviews. The advent of Generative AI offers a unique opportunity to democratize access to high-quality coaching. By harnessing the power of models like Google Gemini, AceIt aims to simulate the nuanced interactions of a human interviewer, providing users with a scalable and effective way to practice.

\section{Problem Statement}
Candidates often struggle with:
\begin{itemize}
    \item \textbf{Lack of Real-time Feedback}: Inability to know immediately where they went wrong in an answer or code snippet.
    \item \textbf{Generic Preparation}: Studying static questions that may not align with their specific resume or target role.
    \item \textbf{Performance Anxiety}: Nervousness during actual interviews due to lack of realistic practice.
    \item \textbf{Technical & Soft Skill Gap}: Difficulty in balancing coding proficiency with effective communication.
\end{itemize}

AceIt addresses these issues by offering a holistic platform that combines technical training with soft skill development.

\section{Objectives}
The primary objectives of AceIt are:
\begin{enumerate}
    \item To develop a \textbf{Simulated Interview Engine} that conducts adaptive interviews using AI, supporting both text and audio inputs.
    \item To implement a \textbf{Smart Coding Arena} with an integrated AI Tutor that provides hints, debugging assistance, and code reviews.
    \item To create a \textbf{Resume Analyzer} that evaluates resumes against job descriptions and provides actionable improvement suggestions.
    \item To provide detailed \textbf{Analytics} on user performance, tracking progress over time.
    \item To build a responsive and intuitive \textbf{User Interface} that mimics professional assessment environments.
\end{enumerate}

\section{Scope}
The project scope encompasses the development of a full-stack web application. The frontend is built using React to ensure a responsive user experience, while the backend utilizes FastAPI for high-performance API handling. The core AI logic is integrated via the Gemini API. The system is designed to be modular, allowing for future extensions such as multi-user mock interviews or integration with corporate hiring portals.
